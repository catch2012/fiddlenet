%
% $Id$ 
%
% $LastChangedDate$ 
% 
% $LastChangedBy$
%

\documentclass[10pt]{unbthesis}
\usepackage{graphicx}
\usepackage{url}
\usepackage{multirow}
\usepackage{setspace}			
\onehalfspacing

\title{CS6735 Research Project: Review of Machine Learning Evaluation
  Methods beyond Accuracy}
\author{Justin Kamerman 3335272}
\date{\today}

\begin{document}
\maketitle
% No chapter numbers
\renewcommand*\thesection{\arabic{section}}

%----------------------------------------
% Abstract
%----------------------------------------
\section*{Abstract}


\section{Introduction}
Key to the field of machine learning is the ability to evaluate a
classifiers performance objectively and extrapolate performance
metrics to predict how a classifier will perform on a previously
unseen inputs. 

Few machine learning algorithms offer the luxury of being analytically
comprehensable e.g. Decision Trees. Such algorithms
allow researchers to analytically predict and compare classifier
performance. In most cases, however, researchers must rely on
empirical methods to evaluate machine learning algorithms. It is
important for the field in general to establish a well defined set of
methods for comparing machine learning methods. If researchers cannot
convince others that they have discovered a better algorithm, or do
convince others with unreliable evaluation techniques, the field risks
excusions down dead-end paths of research or abandonment of those to
potential breakthroughs. 

The ability to evaluate a classifier suitably is not
only important for comparing one classifier or one algorithm to
another but also in the generation classifiers themselves. In
principal, machine learning algorithms involve the search of a
\textit{hypothesis space} for an optimal target function that best
fits the training data. In many cases, this hypothesis space search
involves an explicit comparison of the performance of competing target
function candidates. The ability to pick the best candidate for the
job is at the heart of success, or failure, of a machine learning
algorithm. If such methods cannot converge on an optimal (local or
global) target function, their will be of limited utility in
practice.

In this paper we examine the prevalent evaluation metrics in use by
machine language researchers as well as their origins and shortcomings as
applied in the field. We track the development of new methods which,
in most cases, extend and complement exisiting. We identify domain
specific factors which effect different metrics and should be
considered when selecting a particular evaluation method. In
particular, we look at how the structure and composition of the
training set should effect the choice of evaluation methods.

Essentially any training set is a labelled sample drawn from a
probability distribution of possible instances in the problem
domain. If the set of possible instances is small enough to be
enumerated and labelled, the classification task becomes
trivial. Useful machine learning application are typically given a
sample of labelled instances from which to infer the probability
distribution of a large population. How representative
this sample is of the underlying population is key to how well our
classifiers can hope to perform classifying the rest of the instance
population. The structure of the training set effects different evaluation
metrics differently and, by implication, the target function upon
which they converge. It is important to analyse evaluation methods
in terms of their variability with respect to the structure of the
training set. Certain metrics will be better suited to certain
situations depending on the composition of the training set.

While our discussion will deal with the evaluation of binary
classifiers, the same arguments apply in principal to multiclass
classifiers. In effect, any multiclass classifier can be reduced to a
binary classification problem by decomposition of the hypothesis. 


\section{What's Wrong with Accuracy ?}
Accuracy is the simplest and most intuitive evalaution measure for
classifiers \cite{refworks:43}. Simply stated, accuracy, \(A\), is the
count of how many mistakes classification mistakes made over a finite
set of instances:

\begin{equation}
\label{equ:accuracy}
A = \frac{t_p + t_n}{P + N}
\end{equation}

where \(t_p\) is the number of \textit{true positive} classifications
i.e. postitive training instances that were correctly classified as
positive; \(t_n\) is the number of \textit{true negative}
classifications i.e. negative training instances that were correctly
classified as such. \(P\) is the total number of positive training
example and \(N\) the total number of negative training examples.

Accuracy does not distinguish the types of errors it makes (false
positives vs true negatives). While this is acceptable if the evaluation
data set contains as many examples of both classes (i.e. it is
balanced), for most real-world problems one type of misclassification
error is much more expensive than another \cite{Refworks:45}. Accuracy
maximization also assumes that the class distribution is known for the
target environment. Without knowledge of target class distribution we
cannot be sure that we are actually maximizing accuracy for the
problem domain from which the data set was drawn \cite{RefWorks:45}.


\section{The Confusion Matrix}
The raw data produced by a classification scheme during testing are
counts of the correct and incorrect classifications from each
class. This information is normally displayed in a \textit{confusion
  matrix} showing the difference between the true and predicted
classes for a set of labelled examples as shown in Table
\ref{tab:confusionmatrix}. In Table \ref{tab:confusionmatrix}, \(t_p\)
and \(t_n\) are the number of true positives and true negatives
respectively. \(f_p\) and \(f_n\) are the number of false positives
and false negatives respectively. The row totals, \(C_p\) and \(C_n\),
are the number of predicted positive and negative examples, and the
column totals, \(P\) and \(N\), are the number of actual positive
and negative examples. Although the \textit{confusion matrix} shows
all of the information about the classifier's performance, more
meaningful measures can be extracted from it to illustrate certain
performance criteria. These measures, such as \textit{Accuracy},
\textit{Precision}, and \textit{Recall}, are described in detail in
subsequent sections.

\begin{table}
\centering
  \begin{tabular}{c|c|c|}
    & \multicolumn{2}{|c|}{Actual} \\ \cline{1-3}
 Predicted & +                   & -                  \\ \hline
    +      & \(T_p\)             & \(F_p\)             \\ \hline
    -      & \(F_n\)             & \(T_n\)             \\ \hline
           & \(P = T_p + F_n\)   & \(N = F_p + T_n\)    \\ \cline{2-3}
  \end{tabular}
  \caption{A confusion matrix. \(T_p\) = true positive count, \(F_n\)
  = false negative count, \(F_p\) = false positive count, and \(T_n\)
  is true negative count, \(P\) = actual positive count, \(N\) = actual
  negative count.}
  \label{tab:confusionmatrix}
\end{table}


%----------------------------------------
% Scalar Evaluation Methods
%----------------------------------------
\section{Scalar Evaluation Methods}


\subsection{Precision and Recall}

\begin{equation}
\label{equ:precision}
Precision = \frac{t_p}{t_p + f_p}
\end{equation}


\begin{equation}
\label{equ:recall}
Recall = \frac{t_p}{t_p + f_n}
\end{equation}

%----------------------------------------
% Non-Scalar Evaluation Methods
%----------------------------------------
\section{Non-Scalar Evaluation Methods}
The measures of performance dicussed so far are valid only for one
particular \textit{operating point}, an operating point normally being
chosen so as to minimize the \textit{probability of error}. However,
in general it is not misclasification rate we want to minimize, but
rather \textit{misclassification cost}. \textit{Misclassification
  cost} is usually defined as follows:

\begin{equation}
\label{equ:misclasscost}
Cost = F_p * {C_F}_p + F_n * {C_F}_n
\end{equation}

where \(F_p\) and \(F_n\) are the false postive and false negative
counts respectively, and \({C_F}_p\) and \({C_F}_n\) are the cost of a
false positive and false negative classification respectively. Class
imbalance and asymmetric misclassification costs are related. It has
been suggested that training
set imbalances may be addressed by training a cost sensitive classifier
with the misclassification cost of the minority class greater than
that of the majority class, and one way to make an algorithm cost
sensative is to synthetically imbalance the training set to reflect
the cost ratio \cite{RefWorks:52}. Unfortunately, little work has been
published on either problem \cite{RefWorks:61}.

The specification of a particular set of
misclassification costs or false and positive misclassifcation rates
define a particular \textit{operating point} of a classification system
Non-scalar evaluation methods try to characterize the performance of a
classifier over a range of operating conditions. Although this makes
direct performance comparisons ambiguous, we are more likely to be
able to select the classifier which will perform best in the real
world by applying domain expertise to the results of a non-scalar
evaluation which captures the performance of a classification system over a range
of \textit{operating points}. Despite the benefits of non-scalar
evaluation techniques, there is still a need for a single measure of
classifier performance that is invariant to the decision criteria, and
is easily extended to inlcude cost/benefit analysis
\cite{Refworks:32}. In some cases non-scalar evaluation techniques
have scalar derivatives which allow unqualifed comparison of
classifiers and qualitative comparison of variances between scalar and
non-scalar metrics.

There seems to be growing recognition in the field of the limitations
of scalar evaluation techniques and increased use and development of
non-scalar methods. Below we consider important non-scalar evalauation
techniques:

\subsection{ROC Curves}
The \textit{ROC} curve was first developed by electrical and radar
engineers during World War II for detecting enemy objects in battle
fields and more recently, it has been employed in the medical decision
making community. In addition to being a useful performance graphing
method, \textit{ROC} curves exhibit useful properties that make them
especially useful for domains with skewed class distributions and
unequal classification cost errors \cite{Refworks:39}. 

The curve is a plot of the \textit{sensitivity}, or \textit{true
  positive rate}, vs. \textit{false positive rate} (\(1 −
specificity\) or \(1 - true negative rate\)), for a binary classifier
system as its discrimination threshold is varied (each threshold
defines a classifier).

\begin{equation}
\label{equ:truepositiverate}
TP = \frac{T_p}{T_p + F_n}
\end{equation}

\begin{equation}
\label{equ:falsepositiverate}
TN = \frac{F_p}{T_n + F_p}
\end{equation}

The sample points obtained for various
discrimination thresholds are treated as points on a continuous curve
and normal curve fitting techniques applied. Together with the sample
points obtained during testing, the points \((0,0)\) and \((1,1\) are
included in the curve plot. The lower left point (\(0,0)\) represents
the strategy of never issuing a positive classification. The opposite
stategy of unconditional positive classification is represented by the
upper right hand point \((1,1)\). For classification algorithms which
do not use an explicit decision threshold, like Decision Trees, class
frequencies in the training set can be changed by under or
oversampling to simulate a change in class priors or misclassification
costs. 

\begin{figure}
  \begin{center}
    \includegraphics[width=\textwidth,height=!]{roc}
  \end{center}
  \caption{ROC Curves}
  \label{fig:roc}
\end{figure} 

If the proportion of positive and negative instances changes in a test
set, the \textit{ROC} curve remains the same. This is evident from the
\textit{confusion matrix} shown in Figure
\ref{tab:confusionmatrix}. \textit{ROC} curves are based on
\textit{true positive rate} (\ref{equ:truepositiverate}) and
\textit{false postive rate} (\ref{equ:falsepositiverate}) which are
ratios, normalized for class distribution. Given the relationship
between class distribution and error cost, which we explore below, we
can say that \textit{ROC} curves are invariant with respect to the
operating conditions (class skew and error cost). This is a desirable
property for performance anaylsis since as operating conditions
change, the region of interest on the graph may change, but not the
graph itself.

Figure \ref{fig:roc} shows a sample
\textit{ROC} curve for three classification systems (A, B and C). The
line \(TP=FP\) represents the performance of a random
classifier which exploits no information in the training data. Any
classifier that produces a point below this line performs worse than
random guessing. Such a classifier is said to have useful infomation
but is applying it incorrectly and the sign of the classifier can be
negated to produce a useful \textit{operating point} \cite{Refworks:60}.

\subsection{Comparing Classifiers}
\textit{ROC} curves may be overlaid for comparison. One
point in the diagram is better than another if it is above and to the
left (northwest)
of the other point i.e. has a higher true positive and a lower false positive
rate. Classifiers appearing on the left hand side of the graph near
the \(X\) axis, make positive
classifications only with strong evidence. Such classifiers make few
false positive errors but often have low true positive rates as
well. Classifiers on the upper right hand side of the graph make
positive classifications with weak evidence. Such classifiers will
classifiy nearly all positives correctly but will have a high false
positve rate \cite{Refworks:39}. With respect to Figure \ref{fig:roc},
curve \(A\) represents a dominant classifier system, clearly bettering
systems \(B\) and \(C\), between which there is no clear winner. 

\textit{ROC} curves show the ability of a classifier to rank the
positive instances relative to the negative ones. The scores produced
by a \textit{probablistic} classifier are relative to other instances
only and not properly \textit{calibrated} as true probabilitues
are. For this reason, classifier scores should not be compared across
model classes and conparing model performance at a common theshold is
meaningless \cite{Refworks:39}. 

Provost and Fawcett (\cite{RefWorks:61}) describe techniques for
comparing the performance of a set of learned classifiers. These
techniques are based on the principles of separation of classifier
performance from class and cost distribution and the use of a convex
hull to identify potentially optimal classifier subsets. These
techniques were examined further by Drummond and Holte
(\cite{RefWorks:52}) who describe an alternate projection of the
\textit{ROC} plot in which the expected cost of a classifier is
represented explicitly. These treatments are described in more detail
below: 


\subsubsection{Iso-performance Lines}
The expected cost of applying the classifier represented by a point
\((FP, TP\) in \textit{ROC} space  is:

\begin{equation}
\label{equ:expectedcost}
p(+)(1-TP){C_F}_n + p(-)FP{C_F}_p
\end{equation}

Where \(p(+)\) and \(p(-) = 1 - p(+)\), represent the classes' prior
distribution. Therefore, two points \((FP_1, TP_1\) and \((FP_2,
TP_2\), have the same performance if 

\begin{equation}
  \frac{TP_2 - TP_1}{FP_2 - FP_1} = \frac{{C_F}_p . p(+)}{{C_F}_n . p(-)}
\end{equation}

This equation defines the slope of an \textit{iso-performance
  line}. All classifiers corresponding to points on the line have the
same expected cost. Lines having larger \(TP\)-intercept correspond to
a lower expected cost \cite{RefWorks:61}.

In Figure \ref{fig:roc}, point \(a\) on curve \(A\) represents the
intercept of an \textit{iso-performance line} with gradient
\(3/2\). This corresponds to a scenario in which negatives outnumber
positives by 3 to 2 but false positives and false negatives have equal
cost. Since the classifier on curve \(A\) at point \(a\) is the most
northwest line of slope \(3/2\), it represents the best classifier for
these conditions. Each set of class and cost distribututions defines a
family of \textit{iso-performance} lines.


\subsubsection{\textit{ROC} Convex Hull (ROCCH)}
Given that in the real world, the operating conditions within a
problem domain are seldom known precicely, it is useful to be able to
identify those classifiers which are potentially optimal
\cite{RefWorks:61}. If a collection \textit{ROC} curves are plotted
together, the \textit{convex hull} of the set of points has important
properties. A classifier is potentially optimal if and only if it lies
on the \textit{convex hull} of this set of points in \textit{ROC}
space. Intuitively, the \textit{convex hull} represents the northwest
boundry of the set of points in \textit{ROC} space and, as discussed
above, the locus of the stongest classifiers. An \textit{ROC} curve
with no points on the \textit{convex hull} is not optimal for any
operating conditions. In Figure \ref{fig:roc}, curve \(A\) alone forms
the convex hull of the collection of points so it is potentially
optimal for all operating conditions. Curves \(B\) and \(C\) are
clearly suboptimal for all operating conditions.

As discussed above, a particular operating point defines the slope of
an \textit{iso-performance line} and the \textit{iso-performance line}
with the lowest expected cost is that with the highest
\(TP\)-intercept. It is easy to see that optimal performance for these
operating conditions is achieved by the classifier on the
\textit{ROCCH}, tangent to this \textit{iso-performance line} line. In
Figure \ref{fig:roc} this corresponds to classifier \(a\).


\subsubsection{Area Under the \textit{ROC} Curve (AUC)}
In order evalaute the \textit{ROC} curve as a whole i.e. extract a
single distinguishable scalar metric, the area under the \textit{ROC}
curve, \textit{AUC} seems to exhibit desirable
properties. \textit{AUC} represents the probability that a randomly
chosen positive example is correctly ranked with greater suspicion
than a randomly chosen negative example. \cite{Refworks:32} found good
agreement between \textit{accuracy} and \textit{AUC} with respect to
performance ranking as well as the following desirable properties:

\begin{itemize}
\item Increased sensitivity in the Analysis of Variance (ANOVA) tests.
\item It is not dependent on the decision threshold chosen.
\item It is invariant to prior class probabilities.
\end{itemize}

The area under the \textit{ROC} curve can be adequately approximated
using a trapezoid rule.

It is not possible with a single scalar metric to capture
entirely the performance of a classification system over a range of
operating conditions. It is possible for a high \textit{AUC}
classification system to perform worse than a low
\textit{AUC}. However, in practice the \textit{AUC} is a stron g
predictor of performance and is often used when a general measure of
predictiveness is required \cite{RefWorks:39}.


\subsubsection{Cost Curves}
Drummond and Holte (\cite{RefWorks:52} describe a technique called
\textit{cost curves} to facilitate quantifying the performance
difference between two \textit{ROC} curves. In \textit{ROC} space, the
performance difference between two classifiers is not the Euclidian
distance normal to the lower curve but weighted linear function
incorporating the operating conditions. They argue that while it
is possible to obtain this information from \textit{ROC} curves
directly, it is not trivial. A \textit{cost curve} is a tranform of
the \textit{ROC} space, plotting a probability cost function (Equation
\ref{equ:probcostfunc}) on the \(x\)-axis, against the expected cost
normalized with respect to the cost incurred when every example is
incorrectly classified. 

\begin{equation}
\label{equ:probcostfunc}
\frac{p(+){C_F}_n}{p(+){C_F}_n + p(-){C_F}_p}
\end{equation}

Each point in the \textit{cost curve} corresponds to an
\textit{iso-performance line} in \textit{ROC} space. The probability
cost function of the \textit{cost curve} governing the slope of the
\textit{iso-performance line} and the normalized expected cost, the
\(TP\) intercept. The distance between two cost curves directly
indicates the performance between the two. 

\textit{Cost curves} are no more powerful than \textit{ROC} curves,
offering only an alternative representation of the results of
performance testing that explicitly represents cost differences
between classifiers. While this representation is useful, it is this
authors opinion that its primary motivation is the explicit
quantification of performance differentials. As discussed previously,
performance evaluation is a statistical estimation exercise and should
be used qualitatively to select one classifier as optimal over
another, not to quantify said performance differential.

%----------------------------------------
% Statistical Techniques
%----------------------------------------
\section{Statistical Techniques}
We assume that the true distribution of examples to which a classifier
will be applied is not known in advance. To make an informed choice,
performance must be estimated using the data available
\cite{RefWorks:45}. Application of statistical techniques is
becoming more widespread in the field of machine learning and serve to
rigorize the results of machine learning experimatation and bring it
 in line with more established fields of diagnostics and
classification. Below we discuss some of the techniques employed in
machine learning evaluation papers:

\subsection{Sub-sampling}
Ultimately all machine learning evaluation techniques use a finite set
of training samples to project the performance characteristics of a
classifier over unseen data. It is known that single train and test
partitions are not reliable estimators of true error rate of a
classification scheme on a limited data set \cite{RefWorks:57}. In
addition to selecting an appropriate performance metric, it important
to employ some random sub-sampling scheme to minimize any estimation
bias. \textit{Cross-validation} is a common sub-sampling technique
used in machine learning evaluation.

Scalar metrics obtained from sub-sampling are easily combined into
means and standard deviations. Non-scalar metrics are not so easily
combined and the best method depends on the metric itself as well as
the specific goals of the study. During a
study by Bradley (\cite{RefWorks:32}) using \textit{ROC} curves, it was found
that averaging raw data (frequencies of true and false positives) from
\textit{cross-validation} to produce a \textit{pooled} plots,
depressed the combined index of accuracy, AUC. In the same study,
Bradley selected a technique known as \textit{pooling} to combine the
results of cross-validation to generate ROC curves. Foster et al
(\cite{RefWorks:45} used an alternative methodology, called
\textit{averaging}, after considering Bradley's methods inappropriate
for their requirements.

\subsection{Analysis of Variance}
Analysis of Variance (ANOVA) is a statistical technique to
determine whether or not the means of several groups are equal. In
machine learning it has been used to determine if the results of
performance tests of different classifiers are statisitically
significant.

Studies conducted by \cite{RefWorks:38} examined
the use of ANOVA to determine the significance of means of
\textit{AUC} in Monte Carlo experiments for a
variety of machine learning algorithms. It was indicated that since
ANOVA does not estimate case-sample variance, it tends to be
optimistic, especially for lasrge samples. While study was based on
normal distributions, and therefore not generally applicable, it does
suggest that problem domain specifics should be taken into account
when using ANOVA.

\subsection{Bootstrapping}


\subsection{Confidence Intervals}
Even with the use of some subsampling technique, it is important to
quantify as far as possible the level of confidence that such
approximations offer in favouring one target function or classifier
over another.  We must be able to detect whether test results
indicating that one classifier is superior to another are
statisitaclly significant. 

A common technique used to describe the uncertainity associated with
an estimate is to use \textit{confidence interval} estimates.


%----------------------------------------
% Benchmark Data Sets
%----------------------------------------
\section{The Real World}
While the selection of appropriate metrics and statistical techniques
are important in evaluating and selecting classifiers, it is important
to remember that our application cannot be expected to perform well in
the real world if the training data does not adequately represent the
problem domain. Very few systems generated with machine learning have
been deployed and used in the field for a sizeable period of time
\cite{RefWorks:46}. Part of the reason for this is that the users of
the system and domain experts must be convinced that the system will
perform well with actual data. Real world data from the field can be
surprisingly dirty \cite{RefWorks:58} and historical data used for
training may have been cleaned. It is common practise to use benchmark
data sets like those collected in the Irvine repository
\cite{RefWorks:59} for performance evaluation of machine learning
algorithms. While this practice provides importatnt insights, it is
wrong to judge, on their basis, an algorithm's practical adequacy
\cite{RefWorks:46}.

Comprehensibility goes a long way to building confidence among
stakeholders of a classifiers capabilities, however as mentioned
previously, few machine learning models offer much in this respect.

\section{Conclusions}


%----------------------------------------
% Bibliography
%----------------------------------------
% changes default name Bibliography to References
\renewcommand{\bibname}{References}
\bibliographystyle{IEEEtran}
\bibliography{IEEEabrv,bibliography}

%----------------------------------------
\end{document}
