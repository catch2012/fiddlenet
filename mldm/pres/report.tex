%
% $Id$ 
%
% $LastChangedDate$ 
% 
% $LastChangedBy$
%

\documentclass[10pt]{unbthesis}
\usepackage{graphicx}
\usepackage{url}
\usepackage{multirow}
\usepackage{setspace}			
\onehalfspacing

\title{CS6735 Research Project: Review of Machine Learning Evaluation
  Methods beyond Accuracy}
\author{Justin Kamerman 3335272}
\date{\today}

\begin{document}
\maketitle
% No chapter numbers
\renewcommand*\thesection{\arabic{section}}

%----------------------------------------
% Abstract
%----------------------------------------
\section*{Abstract}


\section*{Introduction}
Key to the field of machine learning is the ability to evaluate a
classifiers performance objectively and extrapolate performance
metrics to predict how a classifier will perform on a previously
unseen inputs. 

Few machine learning algorithms offer the luxury of being analytically
comprehensable e.g. Decision Trees. Such algorithms
allow researchers to analytically predict and compare classifier
performance. In most cases, however, researchers must rely on
empirical methods to evaluate machine learning algorithms. It is
important for the field in general to establish a well defined set of
methods for comparing machine learning methods. If researchers cannot
convince others that they have discovered a better algorithm, or do
convince others with unreliable evaluation techniques, the field risks
excusions down dead-end paths of research or abandonment of those to
potential breakthroughs. 

The ability to evaluate a classifier suitably is not
only important for comparing one classifier or one algorithm to
another but also in the generation classifiers themselves. In
principal, machine learning algorithms involve the search of a
\textit{hypothesis space} for an optimal target function that best
fits the training data. In many cases, this hypothesis space search
involves an explicit comparison of the performance of competing target
function candidates. The ability to pick the best candidate for the
job is at the heart of success, or failure, of a machine learning
algorithm. If such methods cannot converge on an optimal (local or
global) target function, their will be of limited utility in
practice.

In this paper we examine the prevalent evaluation metrics in use by
machine language researchers as well as their origins and shortcomings as
applied in the field. We track the development of new methods which,
in most cases, extend and complement exisiting. We identify domain
specific factors which effect different metrics and should be
considered when selecting a particular evaluation method. In
particular, we look at how the structure and composition of the
training set should effect the choice of evaluation methods.

Essentially any training set is a labelled sample drawn from a
probability distribution of possible instances in the problem
domain. If the set of possible instances is small enough to be
enumerated and labelled, the classification task becomes
trivial. Useful machine learning application are faced with...

While our discussion will deal with the evaluation of binary
classifiers, the same arguments apply in principal to multiclass
classifiers. In effect, any multiclass classifier can be reduced to a
binary classification problem by decomposition of the hypothesis. 


\section*{What's Wrong with Accuracy ?}
Accuracy is the simplest and most intuitive evalaution measure for
classifiers \cite{refworks:43}. Simply stated, accuracy, \(A\), is the
count of how many mistakes classification mistakes made over a finite
set of instances:

\begin{equation}
\label{equ:accuracy}
A = \frac{t_p + t_n}{P + N}
\end{equation}

where \(t_p\) is the number of \textit{true positive} classifications
i.e. postitive training instances that were correctly classified as
positive; \(t_n\) is the number of \textit{true negative}
classifications i.e. negative training instances that were correctly
classified as such. \(P\) is the total number of positive training
example and \(N\) the total number of negative training examples.

Essentially any training set is a labelled sample drawn from a
probability distribution of possible instances in the problem
domain. If the set of possible instances is small enough to be
enumerated and labelled, the classification task becomes
trivial. Useful machine learning application are typically given a
sample of labelled instances from which to infer the probability
distribution of a large population. How representative
this sample is of the underlying population is key to how well our
classifiers can hope to perform classifying the rest of the instance
population. The structure of the training set effects different evaluation
metrics differently and, by implication, the target function upon
which they converge. It is important to analyse any evaluation method
in terms of its variability with respect to the structure of the
training set. Certain metrics will be better suited to certain
situations depending on the composition of the training set.

Accuracy does not distinguish the types of errors it makes (false
positives vs true negatives). While this is acceptable if the evaluation
data set contains as many examples of both classes (i.e. it is
balanced)...

Classification Cost...

\section*{The Confusion Matrix}
The raw data produced by a classification scheme during testing are
counts of the correct and incorrect classifications from each
class. This information is normally displayed in a \textit{confusion
  matrix} showing the difference between the true and predicted
classes for a set of labelled examples as shown in Table
\ref{tab:confusionmatrix}. In Table \ref{tab:confusionmatrix}, \(t_p\)
and \(t_n\) are the number of true positives and true negatives
respectively. \(f_p\) and \(f_n\) are the number of false positives
and false negatives respectively. The row totals, \(C_p\) and \(C_n\),
are the number of truely positive and negative examples, and the
column totals, \(P\) and \(N\), are the number of predicted positive
and negative examples. Allthough the \textit{confusion matrix} shows
all of the information about the classifier's performance, more
meaningful measures can be extracted from it to illustrate certain
performance criteria. These measures, such as \textit{Accuracy},
\textit{Precision}, and \textit{Recall}, are described in detail in
subsequent sections.

\begin{table}
\centering
  \begin{tabular}{c|c|c|c}
    & \multicolumn{2}{|c|}{Actual} \\ \cline{1-3}
    Predicted & + & - & \\ \hline
    + & \(T_p\)           & \(F_p\)           & \(C_p = T_p + F_p\)  \\ \hline
    - & \(F_n\)           & \(T_n\)           & \(C_n = F_n + T_n\)  \\ \hline
    & \(P = T_n + F_n\) & \(N = F_p + T_n\) \\ \cline{2-3}
  \end{tabular}
  \caption{A confusion matrix. \(T_p\) = true positive count, \(F_n\)
  = false negative count, \(F_p\) = false positive count, and \(T_n\)
  is true negative count.}
  \label{tab:confusionmatrix}
\end{table}


%----------------------------------------
% Scalar Evaluation Methods
%----------------------------------------
\section*{Scalar Evaluation Methods}


\subsection*{Precision and Recall}

\begin{equation}
\label{equ:precision}
Precision = \frac{t_p}{t_p + f_p}
\end{equation}


\begin{equation}
\label{equ:recall}
Recall = \frac{t_p}{t_p + f_n}
\end{equation}

%----------------------------------------
% Non-Scalar Evaluation Methods
%----------------------------------------
\section*{Non-Scalar Evaluation Methods}
The measures of performance dicussed so far are valid only for one
particular \textit{operating point}, an operating point normally being
chosen so as to minimize the \textit{probability of error}. However,
in general it is not misclasification rate we want to minimize, but
rather \textit{misclassification cost}. \textit{Misclassification
  cost} is usually defined as follows:

\begin{equation}
\label{equ:misclasscost}
Cost = F_p * {C_F}_p + F_n * {C_F}_n
\end{equation}

where \(F_p\) and \(F_n\) are the false postive and false negative
counts respectively, and \({C_F}_p\) and \({C_F}_n\) are the cost of a
false positive and false negative classification respectively.
Unfortunately, we rarely know the individual misclassification costs
are and so system performance is often specified in terms of the
required false positive an false negative rates. The specification of
a particular set of misclassification costs or false and positive
misclassifcation rates define a particular \textit{operating point} of
a classifier.

Non-scalar evaluation methods try to characterize the performance of a classifier
over a range of operating conditions. Although this makes direct
performance comparisons ambiguous, we are more likely to be able
to select the classifier which will perform best in the real world by
applying domain expertise to the results of a non-scalar evaluation
which captures the performance of a classifier over a range of
\textit{operating points}. Despite the benefits of non-scalar
evaluation techniques, there is still a need for a single measure of
classifier performance that is invariant to the decision criteria, and
is easily extended to inlcude cost/benefit analysis
\cite{Refworks:32}. In some cases non-scalar evaluation
techniques have scalar derivatives which allow unqualifed comparison
of classifiers and qualitative comparison of variances between scalar
and non-scalar metrics.

There seems to be growing recognition in the field of the limitations
of scalar evaluation techniques and increased use and development of
non-scalar methods. Below we consider important non-scalar evalauation
techniques:

\subsection*{ROC Curves}
The \textit{ROC} curve was first developed by electrical and radar
engineers during World War II for detecting enemy objects in battle
fields. The curve is a plot of the \textit{sensitivity}, or
\textit{true positive rate}, vs. \textit{false positive rate} (\(1 −
specificity\) or \(1 - true negative rate\)), for a binary classifier
system as its discrimination threshold is varied. \textit{ROC}
analysis provides tools to select possibly optimal models and to
discard suboptimal ones independently from (and prior to specifying)
the cost context or the class distribution.

The use of ROC curves to evaluate machine learning algorithms was
first discussed by \cite{RefWorks:32}.




\begin{figure}
  \begin{center}
	\includegraphics[width=\textwidth,height=!]{roc}
  \end{center}
  \caption{ROC Curve}
  \label{fig:roc}
\end{figure} 


%----------------------------------------
% Statistical Techniques
%----------------------------------------
\section*{Statistical Techniques}
Ultimately all machine learning evaluation techniques use a finite set
of training samples to project the performance characteristics of
a classifier over unseen data. It is known that single train and
test partitions are not reliable estimators of true error rate of a
classification scheme on a limited data set \cite{RefWorks:57}. In
addition to selecting an appropriate performance metric, it important
to employ some random sub-sampling scheme to minimize any estimation
bias. \textit{Cross-validation} is a common sub-sampling technique used in machine
learning evaluation. 

Even with the use of some subsampling technique, it is important to
quantify as far as possible the level of confidence that such
approximations offer in favouring one target function or classifier
over another.  We must be able to detect whether test results
indicating that one classifier is superior to another is statisitaclly
significant. Application of statistical techniques is becoming more
widespread in the field of machine learning and serve to rigorize the results of machine learning
experimatation and bring it in line with more established fields of
diagnostics and classification.

A common technique used to describe the uncertainity associated with
an estimate is to use \textit{confidence interval} estimates.




\section*{Summary}


%----------------------------------------
% Bibliography
%----------------------------------------
% changes default name Bibliography to References
\renewcommand{\bibname}{References}
\bibliographystyle{IEEEtran}
\bibliography{IEEEabrv,bibliography}

%----------------------------------------
\end{document}
